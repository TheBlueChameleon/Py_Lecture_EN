% =========================================================================== %

\begin{frame}[t,plain]
\titlepage
\end{frame}

% =========================================================================== %

\begin{frame}[fragile]{Converting Data Types}
%
\begin{itemize}
\item Variables store values
\item We want to do computations with these values
\item Data types affect the computation (\inPy{"1" + "2"} vs. \inPy{1 + 2})
\item Data types can be incompatible (\inPy{"1" + 2} causes an error message)
\item Data types can be used as functions: \inPy{float(1)} \thus~ \inPy{1.0}
\end{itemize}
%
\begin{codebox}[Code: Data Type Conversion, width=.5\linewidth, nobeforeafter, equal height group = grpToString]
\begin{minted}[linenos, fontsize=\scriptsize]{python}
text = "The result is: "
result = ((1 + 1j) * (7.5 / 19 + 1)) ** 2

text += str(result)

print(text)
\end{minted}
\end{codebox}
%
\begin{cmdbox}[Output: Data Type Conversion, width=.49\linewidth, nobeforeafter, equal height group = grpToString]
\begin{minted}[fontsize=\scriptsize]{text}
The result is: 3.890581717451524j
\end{minted}
\end{cmdbox}
%
\end{frame}

% =========================================================================== %

\begin{frame}[fragile]{Converting Data Types}
%
\begin{itemize}
\item Loss of information possible
	\begin{itemize}
	\item \inPy{int(1.2)} \thus~ \inPy{1}
	\end{itemize}
\item Conversion not always possible
	\begin{itemize}
	\item \inPy{int("1")} -- okay, result \inPy{1}
	\item \inPy{float("1.0")} -- okay, result \inPy{1.0}
	\item  \inPy{int("1.0")} -- Error message
	\end{itemize}
\end{itemize}
%
\begin{cmdbox}[Error Message -- \texttt{int("1.0")}]
\begin{minted}[fontsize=\scriptsize]{text}
Traceback (most recent call last):
  File "<stdin>", line 1, in <module>
ValueError: invalid literal for int() with base 10: '1.0'
\end{minted}
\end{cmdbox}
%
\end{frame}

% =========================================================================== %

\begin{frame}[fragile]{Generating Formatted Strings}
%
\begin{itemize}
\item Output in tabular format
\item Number of decimals not fixed
\item Solution: \emph{Format Strings}: specify format in string
\item Prefix \inPy{f}, thereafter normal string in quotes (\inPy{"} oder \inPy{'})
\item Expression to be convertet to string in \{curly brackets\}
\item Optionally: colon (:) and format specifiers
\end{itemize}
%
\begin{codebox}[Code: Format Strings (1), width=.5\linewidth, nobeforeafter, equal height group = grpFormatString1]
\begin{minted}[linenos, fontsize=\scriptsize]{python}
x = 7 / 3
text = f"x = {x}"

print(text)
\end{minted}
\end{codebox}
%
\begin{cmdbox}[Output: Format Strings (1), width=.49\linewidth, nobeforeafter, equal height group = grpFormatString1]
\begin{minted}[fontsize=\scriptsize]{text}
x = 2.3333333333333335
\end{minted}
\end{cmdbox}
%
\end{frame}

% =========================================================================== %

\begin{frame}[fragile]{Generating Formatted Strings: Column Width}
%
\begin{itemize}
\item Format specifiers: can be a single number
\item Specifies the number of characters to be used for representing the expression
\item Unused characters will be padded with whitespaces
\item Excess characters: format specifier is ignored
\item \emph{No} whitespaces
\end{itemize}
%
%\vspace{-10pt}
\begin{codebox}[Code: Format Strings (2), width=.49\linewidth, nobeforeafter, equal height group = grpFormatString2]
\begin{minted}[linenos, fontsize=\scriptsize]{python}
name1 = "Dusky"
score1 = 9001
name2 = "Joe"
score2 = 666
print("Highscore")
print(f"{'Python':20}: {123456:5}")
print(f"{name1:20}: {score1:5}")
print(f"{name2:20}: {score2:5}")
\end{minted}
\end{codebox}
%
\begin{cmdbox}[Output: Format Strings (2), width=.49\linewidth, nobeforeafter, equal height group = grpFormatString2]
\begin{minted}[fontsize=\scriptsize]{text}
Highscore
Python              : 123456
Dusky               :  9001
Joe                 :   666
\end{minted}
\end{cmdbox}
%
\end{frame}

% =========================================================================== %

\begin{frame}[fragile]{Format Strings: Alignment and Clipping}
%
\begin{itemize}
\item \texttt{<}, \texttt{>}, \inPy{^}: Alignment left, right, centered
\item \texttt{.number}: Limit output to \texttt{number} characters (truncate)
\item whitespace or plus: reserve space for a sign
\item \texttt{0}: show leading zeros
\end{itemize}
%
\begin{codebox}[Code: Format Strings (3), width=.49\linewidth, nobeforeafter, equal height group = grpFormatString3]
\begin{minted}[linenos, fontsize=\scriptsize]{python}
text = "sample"
value = 123
print( f"|{text:15}| |{value:15}|" )
print(f"|{text:<15}| |{value:<15}|")
print(f"|{text:>15}| |{value:>15}|")
print(f"|{text:^15}| |{value:^15}|")
\end{minted}
\end{codebox}
%
\begin{cmdbox}[Output: Format Strings (3), width=.49\linewidth, nobeforeafter, equal height group = grpFormatString3]
\begin{minted}[fontsize=\scriptsize]{text}
|sample         | |            123|
|sample         | |123            |
|         sample| |            123|
|    sample     | |      123      |
\end{minted}
\end{cmdbox}
%
\end{frame}

% =========================================================================== %

\begin{frame}[fragile]{Format Strings: Floating Point Numbers}
%
\begin{itemize}
\item Format Specifier \texttt{f}
\item \texttt{.number}: limit to \texttt{number} characters (truncate)
\item \texttt{total.decimal} -- use \texttt{total} characters, of which \texttt{decimal} after the point
\item Attention: Decimal point is counted as well
\item Can be combined with other characters
\end{itemize}
%
\begin{codebox}[Code: Format Strings (4), width=.49\linewidth, nobeforeafter, equal height group = grpFormatString4]
\begin{minted}[linenos, fontsize=\scriptsize]{python}
num = 1.2
print(f"|{num}|")
print(f"|{num:f}|")
print(f"|{num:6.2f}|")
print(f"|{num:<6.1f}|")
print(f"|{num:06.2f}|")
print(f"|{num:+06.2f}|")
print(f"|{num: 06.2f}|")
\end{minted}
\end{codebox}
%
\begin{cmdbox}[Output: Format Strings (4), width=.49\linewidth, nobeforeafter, equal height group = grpFormatString4]
\begin{minted}[fontsize=\scriptsize]{text}
|1.2|
|1.200000|
|  1.20|
|1.2   |
|001.20|
|+01.20|
| 01.20|
\end{minted}
\end{cmdbox}
%
\end{frame}

% =========================================================================== %

\begin{frame}{The Main Concern}
%
\begin{center}
\begin{Large}
\emph{Do I have to know all of this by heart?}
\end{Large}
\end{center}
%
\begin{itemize}
\item Nope.
\item You can use the lecture notes, slides, even the internet at any time.
\item Know that these options do exist
\item Be able to read this
\item Be able to use this with the aid of literature
\item Given enough time you \emph{will} know these by heart
\item But: even after 20 years of coding I have to look up some details
\end{itemize}
%
\end{frame}

% =========================================================================== %

\begin{frame}{Chapter 2}
%
\begin{itemize}
\item  User Input and Conditions
	\begin{itemize}
	\item User input -- \inPy{input}
	\item Booleans (truth values), comparison operators and logical operators
	\item \inPy{if}-compounds
	\item Ternary Operator
	\end{itemize}
\end{itemize}
%
\end{frame}

% =========================================================================== %

\begin{frame}[fragile]{User Input}
%
\begin{itemize}
\item Command \inPy{input}: Read data from the keyboard
\item Optional parameter \enquote{prompt}: Text to be shown pre-input, \zB \enquote{question} to the user
\item Result is \emph{returned} as a string
\item[\Thus] Type conversion might be necessary
\item User can (and will!) enter utter BS
\end{itemize}
%
\begin{codebox}[Code: \texttt{input}, width=.49\linewidth, nobeforeafter, equal height group = grpInput]
\begin{minted}[linenos, fontsize=\scriptsize]{python}
inText = input(
    "Please provide a number: ")
inNumber = float(inText)
print("The square of your number is",
      inNumber ** 2)
\end{minted}
\end{codebox}
%
\begin{cmdbox}[Output: \texttt{input}, width=.49\linewidth, nobeforeafter, equal height group = grpInput]
\begin{minted}[fontsize=\scriptsize]{text}
Please provide a number: 1.5
The square of your number is 2.25
\end{minted}
\end{cmdbox}
%
\end{frame}

% =========================================================================== %

\begin{frame}[fragile]{Truth Values}
%
\begin{itemize}
\item Data type \inPy{bool} (aka \emph{Boolean}, after mathematician George Bool):\\
	Idea Yes (\inPy{True}) or No (\inPy{False})
\item Result of comparisons: \inPy{1 + 5 * 8 + 1 == 42} evaluates to \inPy{True}
\item Can be stored in variables: \inPy{truth = (1 + 5 * 8 + 1 == 42)}
\item Internally: Numbers \inPy{1} and \inPy{0} with special interpretation\\
	\inPy{True + True == 2}
\end{itemize}
%
\newcolumntype{C}{>{         \centering\arraybackslash} p{.245\linewidth}}
\newcolumntype{O}{>{\ttfamily\centering\arraybackslash} p{.200\linewidth}}

\rowcolors{1}{tabhighlight}{white}
\begin{center}
\begin{tabularx}
	{\linewidth}
	{CO|CO}
\toprule[1pt]
	
	\textbf{Comparison}  & \normalfont \textbf{Symbol}  &  
	\textbf{Comparison}  & \normalfont \textbf{Symbol}
\tabcrlf

	Equity        & ==                   &  Inequity         & != \textrm{or} <>\\
	Less Than     & <                    &  Greater Than     & >  \\
	Less or Equal & <=                   &  Greater or Equal & >= \\
	
\bottomrule[1pt]
\end{tabularx}
\end{center}
%
\end{frame}

% =========================================================================== %

\begin{frame}[fragile]
%
\begin{minipage}[t]{.49\linewidth}
\begin{Large}
Logical Operators
\vspace{6pt}
\end{Large}
\begin{itemize}
\item \enquote{Linking} truth values
\item \inPy{and}: both expressions satisfied
\item \inPy{or}: at least one expr. satisfied
\item Apply on \inPy{bool}s directly (example to the right) ...
\item ... or in composite expressions:\\
	\inPy{good = (T > 20) and (T < 25)}
\item Other languages sometimes implement XOR (exclusive or).\\
	Python: Use inequity operator with \inPy{bool}s (\texttt{!=})
\end{itemize}
\end{minipage}
%
\begin{minipage}[t]{.49\linewidth}
\phantom{x}
\begin{codebox}[Code: Linking with \texttt{and}]
\begin{minted}[linenos, fontsize=\scriptsize]{python}
temperature = float(input(
    "Current temperature? "
))
warmEnough = temperature > 20
coldEnough = temperature < 25
pleasantTemperature = warmEnough and \
                      coldEnough
print("'It's pleasantly warm' is",
      pleasantTemperature)
\end{minted}
\end{codebox}
\begin{cmdbox}[Output: Linking with \texttt{and}]
\begin{minted}[fontsize=\scriptsize]{text}
Current temperature? 18
'It's pleasantly warm' is False
\end{minted}
\end{cmdbox}
\end{minipage}
%
\end{frame}

% =========================================================================== %

\begin{frame}[fragile]{Conditional Execution: \inPy{if}-Compounds}
%
\begin{minipage}[t]{.49\linewidth}
\begin{itemize}
\item If-then-else-structure: execute code only under condition
\item Condition: truth value
\item Code: anything we already know (and will learn later).
	\begin{itemize}
	\item at least one line
	\item arbitrarily many lines
	\item nested \inPy{if}s possible
	\end{itemize}
\item Indentation marks what is part of the block
\item After the \inPy{if}-compound: remove indentation
\end{itemize}
\end{minipage}
%
\begin{minipage}[t]{.49\linewidth}
\phantom{x}
\begin{codebox}[Syntax: \texttt{if} (1)]
\begin{minted}[fontsize=\scriptsize]{python}
if truthValue :
    statements
    ...
\end{minted}
\end{codebox}
%
\begin{codebox}[Syntax: \texttt{if} (2)]
\begin{minted}[fontsize=\scriptsize]{python}
if truthValue :
    statements
    ...
else :
    statements
    ...
\end{minted}
\end{codebox}
\end{minipage}
%
\end{frame}

% =========================================================================== %

\begin{frame}{Indentation}
%
\begin{itemize}
\item Whitespaces \emph{or} tabulators
\item No mixing (in the same document)
\item Same number of whitespaces per layer for the entire code
\end{itemize}
%
\begin{hintbox}
Recommendation by the Python-devs: four whitespaces\\
Many code editors (like Spyder) automatically interpret Tab as four whitespaces and even recognize block statements!
\end{hintbox}
%
\end{frame}

% =========================================================================== %

\begin{frame}[fragile]
%
\begin{codebox}[Example]
\begin{minted}[linenos, fontsize=\scriptsize]{python}
temperature = float(input("Please state the current temperature in Celsius: "))

if (temperature > 20) and (temperature < 25) :
    print("It's pleasantly warm.")
else :
    print("It's either too hot or too cold.")
\end{minted}
\end{codebox}
%
\begin{cmdbox}[Output 1]
\begin{minted}[fontsize=\scriptsize]{text}
Please state the current temperature in Celsius? 23
It's pleasantly warm
\end{minted}
\end{cmdbox}
%
\begin{cmdbox}[Output 2]
\begin{minted}[fontsize=\scriptsize]{text}
Please state the current temperature in Celsius? 18
It's either too hot or too cold
\end{minted}
\end{cmdbox}
%
\end{frame}

% =========================================================================== %

\begin{frame}[fragile]{\inPy{if} With Multiple Conditions}
%
\begin{minipage}[t]{.49\linewidth}
\begin{itemize}
\item First condition is checked first.
\item If (and only if!) not satisfied: first \inPy{elif}; then second \inPy{elif}, ...
\item If several conditions satisfied: \emph{only first block} is executed!
\item There may be arbitrarily many \inPy{elif}-Blocks
\item \inPy{else} can be used, too, optionally
\end{itemize}
\end{minipage}
%
\begin{minipage}[t]{.49\linewidth}
\phantom{x}
\begin{codebox}[Syntax: \texttt{if} (3)]
\begin{minted}[fontsize=\scriptsize]{python}
if truthValue1 :
    statements
    ...
elif truthValue2 :
   statements
   ...
else :
    statements
    ...
\end{minted}
\end{codebox}
\end{minipage}
%
\end{frame}

% =========================================================================== %

\begin{frame}[fragile]
%
\begin{codebox}[Example: \texttt{elif}]
\begin{minted}[linenos, fontsize=\scriptsize]{python}
temperature = float(input("Please state the current temperature in Celsius: "))
if temperature < 20 :
    print("It's too cold.")
elif temperature > 25 :
    print("It's too hot")
else :
    print("The temperature is pleasant.")
\end{minted}
\end{codebox}
%
\begin{warnbox}[Erroneous Code: \texttt{elif}, leftupper=6mm]
\begin{minted}[linenos, fontsize=\scriptsize]{python}
i = 23

if i < 25 :
    print("This number is less than 25.")
elif i > 20 :
    print("This number is bigger than 20.")
\end{minted}
\end{warnbox}
%
\end{frame}

% =========================================================================== %

\begin{frame}[fragile]
%
\begin{codebox}[Example: Nested \texttt{if}-Compounds]
\begin{minted}[linenos, fontsize=\scriptsize]{python}
i = int(input("Please provide an integer: "))

if i % 2 == 0 :
    if   i > 100 :
        print(i, "is a big even number.")
    elif i >   0 :
        print(i, "is a smalll even number.")
    else :
        print(i, "is a negative even number.")
else :
    print(i "is odd.")
\end{minted}
\end{codebox}
%
\end{frame}

% =========================================================================== %

\begin{frame}[fragile]
%
\begin{warnbox}[Redundancy with \texttt{if} and Logical Operators: \texttt{elif}, leftupper=6mm]
\begin{minted}[linenos, fontsize=\scriptsize]{python}
i = int(input("Please provide an integer: "))

if   i % 2 == 0 and i > 100 :
    print(i, "is a big even number.")
elif i % 2 == 0 and i <   0 :
    print(i, "is a smalll even number.")
elif i % 2 == 0 :
    print(i, "is a negative even number.")
else :
    print(i "is odd.")
\end{minted}
\end{warnbox}
%
\begin{itemize}
\item Triple condition: \texttt{i \% 2}
\item Typos and forgotten lines in editing code later
\item Diminished legibility: \inPy{else}
\end{itemize}
%
\end{frame}

% =========================================================================== %

\begin{frame}[fragile]
%
\begin{tcbraster}[raster columns=2,
                  raster equal height,
                  nobeforeafter,
                  raster column skip=0.5cm]
	\begin{warnbox}[Validity Check: Sequence of \texttt{if}s, leftupper=7mm]
	\begin{minted}[linenos, fontsize=\scriptsize]{python}
playerCount = int(input(
    "How many players? "
))
      
if playerCount < 2 :
    print("Cannot play with", 
          playerCount, 
          "players."
    )
if playerCount > 5) :
    print("Cannot play with", 
          playerCount,
          "players."
    )
	\end{minted}
	\end{warnbox}
%
	\begin{codebox}[Validity Check: Logical \texttt{or}]
	\begin{minted}[linenos, fontsize=\scriptsize]{python}
playerCount = int(input(
    "How many players? "
))
      
if  playerCount < 2 or
    playerCount > 5  :
        print("Cannot play with",
              playerCount,
              "players."
        )
	\end{minted}
	\end{codebox}
\end{tcbraster}
%
\end{frame}

% =========================================================================== %

\begin{frame}
%
\begin{hintbox}[Conclusion]
Question your Structures. Avoid Redundancy. Whenever you would copy and paste: Check if there isn't a better solution.

In bigger projects I write almost each line (that works in the first place) \emph{twice}: Once as a concept and once to keep the code useable in the long run.
\end{hintbox}
%
\end{frame}