% =========================================================================== %

\begin{frame}[t,plain]
\titlepage
\end{frame}

% =========================================================================== %

\begin{frame}[fragile]{Recap}
%
\begin{itemize}
\item Deck: \inPy{list} of \inPy{tuple}s of strings
	\begin{itemize}
	\item \texttt{(suit, value)}
	\end{itemize}
\item Players: \inPy{list} of \inPy{dict}s
	\begin{itemize}
	\item Entries \inPy{'name'}, \inPy{'money'}, \inPy{'score'}
	\end{itemize}
\item Functions generate these lists
\end{itemize}
%
\end{frame}

% =========================================================================== %

\begin{frame}[fragile]{Analysis: The Game}
%
\begin{itemize}
\item Turn-Based Game
	\begin{itemize}
	\item Loops
	\end{itemize}
\item Multiple rounds
	\begin{itemize}
	\item Loops within loops
	\end{itemize}
\item Players choose action
	\begin{itemize}
	\item Interpretation of some \inPy{input}
	\end{itemize}
\item Choices before each round
	\begin{itemize}
	\item Resign (Leave with the money they own)
	\item Play another round and bet some money
	\end{itemize}
\item Choices in each turn
	\begin{itemize}
	\item reste -- draw no more cards
	\item carte -- draw one more card
	\end{itemize}
\end{itemize}
%
\end{frame}

% =========================================================================== %

\begin{frame}[fragile]{Analysis: The Game}
%
\begin{itemize}
\item Different choices imply players have a \emph{state}
	\begin{itemize}
	\item \inPy{'active'} -- default state
	\item \inPy{'resting'} -- after declining another card
	\item \inPy{'retired'} -- after leaving the game
	\end{itemize}
\item Expand our model of a player: Add key \inPy{'state'} to the \inPy{'dict'}
\item Can be done directly in \texttt{registerPlayer}
\item Needs to be changed in other functions
\item You probably just lost \emph{The Game}.
\end{itemize}
%
\end{frame}

% =========================================================================== %

\begin{frame}[fragile]{Analysis: The Game}
%
\begin{itemize}
\item Rounds
	\begin{itemize}
	\item Ask everyone whether they stay
	\item If yes, ask for the money they would bet
	\item Combine: betting 0 money means retire
	\end{itemize}
\item In each round: multiple turns
	\begin{itemize}
	\item Ask: reste/carte
	\item Evaluate: Over 21?
	\item Is there more than one active player? \Thus End of round
	\end{itemize}
\item End of the game: everyone has gone inactive
\item Neither Rounds nor turns: predeterminate count
	\begin{itemize}
	\item[\Thus] \inPy{while} and some condition
	\item Conditions for end of round/turn: count active players
	\item Nontrivial. Should be Function
	\end{itemize}
\end{itemize}
%
\end{frame}

% =========================================================================== %

\begin{frame}[fragile]
%
\begin{codebox}[Update: \texttt{registerPlayers}]
\begin{minted}[fontsize=\scriptsize, linenos]{python}
def registerPlayers() :
    players = []
    notDoneYet = True
    
    while notDoneYet :
        name = input("Please enter the next player's name "
                     "(empty line ends player registration): ")
        if len(name) == 0 :
            notDoneYet = False
            break
        
        money = float( input(f"Please enter {name}'s start money: ") )
        
        players.append(
            {'name' : name, 'money' : money, 'cards' : [], 'state' : 'active'}
        )                                               #  ------- /\ -------
    
    return players
\end{minted}
\end{codebox}
%
\end{frame}

% =========================================================================== %

\begin{frame}[fragile]
%
\begin{codebox}[New Function: \texttt{countActivePlayers}]
\begin{minted}[fontsize=\scriptsize, linenos]{python}
def countActivePlayers(players) :
    reVal = 0
    for player in players :
        if player['state'] == 'active' :
            reVal += 1
    return reVal
\end{minted}
\end{codebox}
%
\begin{hintbox}[Hint: Write out of order]
When writing functions that compute a value: your first line should always be
\mint{python3}{variable = default_value}
Proceed with
\mint{python3}{return variable}
Then, write the stuff in between
\end{hintbox}
%
\end{frame}

% =========================================================================== %

\begin{frame}[fragile]{Synthesis: The Main Game Loop}
%
Write outline first:
%
\begin{codebox}[Main Game Loop]
\begin{minted}[fontsize=\scriptsize, linenos]{python}
while countActivePlayers(players) > 0 :       # rounds
    pool = 0         # the total money everyone has bet
    
    for player in players :
        if player['state'] != 'active' : continue
        pool += handleBets(player)
    
    while countActivePlayers(players) > 1 :   # turns
        for player in players :
            handleDrawCards(player, deck)
    
    handleMoneyGains(players, pool)           # determines winner, also handles ties
    reactivatePlayers(players)                # changes state from 'resting' to 'active'
    kickBrokePlayers(players)
\end{minted}
\end{codebox}
%
\end{frame}

% =========================================================================== %

\begin{frame}
%
\begin{hintbox}[Fill In Required Structures With Minimal Code]
We introduced 5 functions (\texttt{handleBets}, \texttt{handleDrawCards}, \texttt{handleMoneyGains}, \texttt{reactivatePlayers}, \texttt{kickBrokePlayers}). You'll want to test your loop before fleshing them out. Don't write too much code at once -- start out with duds (empty functions).

Attention: You might create an infinite loop!
\end{hintbox}
%
\begin{itemize}
\item Game can only end if all players decide to leave or go bankrupt.
\item Bankrupcy triggers later than player decision
\item[\Thus] Write code for \texttt{handleBets}, leave other functions empty
\end{itemize}
%
\end{frame}

% =========================================================================== %

\begin{frame}[fragile]
%
\begin{codebox}[Functions Called by Main Game Loop (Duds)]
\begin{minted}[fontsize=\scriptsize, linenos]{python}
def handleBets(player) :
    player['state'] = 'retired'
    return 0

def handleDrawCards(player, deck) :
    pass

def handleMoneyGains (players, pool) :
    pass

def reactivatePlayers (players) :
    pass

def kickBrokePlayers (players) :
    pass
\end{minted}
\end{codebox}
%
\end{frame}

% =========================================================================== %

\begin{frame}[fragile]
%
\begin{codebox}[\texttt{handleBets} -- Fleshed Out]
\begin{minted}[fontsize=\scriptsize, linenos]{python}
def handleBets(player) :
    bet = -1
    while bet < 0 :
        print(f"{player['name']}: You have {player['money']} coins. "
               "How much do you bet?")
        bet = float(input("(Type 0 to retire): "))
    
    if bet == 0 :
        player['state'] = 'retired'
    
    player['money'] -= bet
    
    return bet
\end{minted}
\end{codebox}
%
\begin{codebox}[\texttt{handleDrawCards} -- Update for test]
\begin{minted}[fontsize=\scriptsize, linenos]{python}
def handleDrawCards(player) :
    player['state'] = 'retired'
\end{minted}
\end{codebox}
%
\end{frame}

% =========================================================================== %

\begin{frame}[fragile]
%
\begin{codebox}[\texttt{handleDrawCards} -- Fleshed Out]
\begin{minted}[fontsize=\scriptsize, linenos]{python}
def handleDrawCards(player) :
    print(f"{player['name']}: Your hand holds these cards:")
    print(f"{player['cards']}")
    
    noValidAnswer = True
    while noValidAnswer :
        answer = input("Do you want another card (yes/no)? ").upper()
        if answer in ('YES', 'NO') :
            noValidAnswer = False
    
    if answer == 'YES' :
        player['cards'].append( deck.pop() )
        checkLimit(player)     # to be fleshed out!
        
    else :
        player['state'] = 'resting'


def checkLimit(player) :
    pass
\end{minted}
\end{codebox}
%
\end{frame}

% =========================================================================== %

\begin{frame}[fragile]
%
\begin{codebox}[\texttt{checkLimit} -- Fleshed Out]
\begin{minted}[fontsize=\scriptsize, linenos]{python}
def checkLimit(player) :
    total = 0
    for card in player['cards'] :
        total += card[1]     # actually, this doesn't work (yet)
    
    if total > 21 :
        print("You've busted your limit!")
        player['state'] = 'resting'
\end{minted}
\end{codebox}
%
\begin{warnbox}[Major Redesigns]
\inPy{card[1]} is a string. We could write a function that computes the value from the given string. But wait -- didn't we have such a lookup when we first conceived our \texttt{buildDeck} function?

It \emph{might} be easier to re-structure some code. Before every decision you make, take a step back and have a look at the big picture.
\end{warnbox}
%
\end{frame}

% =========================================================================== %

\begin{frame}[fragile]{Analysis: The Big Picture}
%
\begin{itemize}
\item We store each card as a \inPy{tuple} of strings
\item But we also would like the value ready at hand
\item Do we really need the value stored in string form?
	\begin{itemize}
	\item No, can be restored on the fly with existing \texttt{lookupCardName}
	\item Only point where card output happens: in \texttt{handleDrawCards}
	\item[\Thus] Effort to change this manageable
	\item Advantage: Saves on definitions
		\begin{itemize}
		\item We already have a lookup value \thus string
		\item The reverse lookup would effectively be duplicate code
		\item We don't like redundancy
		\end{itemize}
	\end{itemize}
\item Let's go and change stuff!
\end{itemize}
%
\end{frame}

% =========================================================================== %

\begin{frame}[fragile]
%
\begin{codebox}[\texttt{buildDeck} -- Update]
\begin{minted}[fontsize=\scriptsize, linenos]{python}
def buildDeck() :
    deck = []
    
    for deckID in range(6) :
        for suit in ("Hearts", "Diamond", "Club", "Spades") :
            for value in ( range(1, 14) ) :
                deck.append( (suit, value) )
                # previously: deck.append( (suit, lookupCardName(value)) )
    
    random.shuffle(deck)    # look how nice this reads! Like "normal" English!
    return deck
\end{minted}
\end{codebox}
%
\end{frame}

% =========================================================================== %

\begin{frame}[fragile]
%
\begin{codebox}[\texttt{handleDrawCards} -- Update]
\begin{minted}[fontsize=\scriptsize, linenos]{python}
def handleDrawCards(player, deck) :
    print(f"{player['name']}: Your hand holds these cards:")
    # previously: print(f"{player['cards']}")
    for card in player['cards'] :
        print( lookupCardName(card[1]), "of", card[0] )
    
    noValidAnswer = True
    while noValidAnswer :
        answer = input("Do you want another card (yes/no)? ").upper()
        if answer in ('YES', 'NO') :
            noValidAnswer = False
    
    if answer == 'YES' :
        card = deck.pop()               # previously: no output at all
        player['cards'].append( card )
        print("You drew a", card[1], "of", lookupCardName(card[0]) )
        checkLimit(player)
        
    else :
        player['state'] = 'resting'
\end{minted}
\end{codebox}
%
\end{frame}

% =========================================================================== %

\begin{frame}{Take a Step Back and See What We've Accomplished}
%
We almost did it!
\begin{itemize}
\item Representing Cards and Players
\item Defining the Flow of the Game
\item React to User Input
\item Model the principal game mechanic (hand over 21 points)
\end{itemize}

What's left to do?
\begin{itemize}
\item Give Money to winners each round
\item Prepare the next round
\item Treat bankrupt players
\item Show some nice end-of-game scren
\end{itemize}
%
\end{frame}

% =========================================================================== %

\begin{frame}{Analysis/Synthesis: Give Money To Winners}
%
\begin{itemize}
\item Compute total card value for each player's hand
	\begin{itemize}
	\item \inPy{list} with intermediate results
	\end{itemize}
\item Find most valuable hand(s)
	\begin{itemize}
	\item \inPy{list} of indices
	\item Directly, we only get \emph{values}
	\item Use \inPy{enumerate} to get both at the same time
	\item Use \texttt{sort}
	\item Find first hand that has less value than its predecessor \Thus everyone before that belongs to the winners
	\end{itemize}
\item Update player money
	\begin{itemize}
	\item for each winner: \inPy{players[ID]['money'] += pool / len(winners)}
	\end{itemize}
\end{itemize}
%
\end{frame}

% =========================================================================== %

\begin{frame}[fragile]
%
\begin{codebox}[\texttt{handleMoneyGains} -- Fleshed Out]
\begin{minted}[fontsize=\scriptsize, linenos]{python}
def handleMoneyGains(player, pool) :
    winners = []
    for i, player in enumerate(players) :
        total = 0
        for card in player['cards'] :
            total += card[1]
        winners.append( (total, i) )
    
    winners.sort()
    
    lastWinID = 0
    lastWinScore = winners[0][0]
    for winner in winners :
        if winner[0] < lastWinScore :
            break
        lastWinID += 1
    winners = winners[:lastWinID]
    
    for total, ID in winners :
        players[ID]['money'] += pool / lastWinID   # by design, len(winners) == lastWinID
\end{minted}
\end{codebox}
%
\end{frame}

% =========================================================================== %

\begin{frame}
%
\begin{warnbox}[Redundant Code]
Notice how we have written the same code twice?

Our method to compute the total value of a hand already appeared in \texttt{checkLimit}.\\
Better put this into a function of its own.
\end{warnbox}
%
\end{frame}

% =========================================================================== %

\begin{frame}[fragile]
%
\begin{codebox}[Remove Redundancies]
\begin{minted}[fontsize=\scriptsize, linenos]{python}
def getTotalValue(player) :
    total = 0
    for card in player['cards'] :
        total += card[1]
    return total

def checkLimit(player) :
    if getTotalValue(player) > 21 :
        print("You've busted your limit!")
        player['state'] = 'resting'

def handleMoneyGains (players, pool) :
    winners = []
    for i, player in enumerate(players) :
        winners.append( (getTotalValue(player), i) )
    
    winners.sort()
    
    # as before ...
\end{minted}
\end{codebox}
%
\end{frame}

% =========================================================================== %

\begin{frame}[fragile]{Analysis / Synthesis: Preparing the next round}
%
\begin{itemize}
\item Find all resting players and set them back to active
\item Simple \inPy{for} loop
\end{itemize}
%
\begin{codebox}[Preparing The Next Round]
\begin{minted}[fontsize=\scriptsize, linenos]{python}
def reactivatePlayers (players) :
    for player in players :
        if player['state'] == 'resting' :
            player['state'] = 'active'
\end{minted}
\end{codebox}
%
\end{frame}

% =========================================================================== %

\begin{frame}[fragile]{Analysis / Synthesis: Treat Bankrupt Players}
%
\begin{itemize}
\item Find all players with 0 or less money
\item Set their status to \inPy{'retired'}
\item Simple \inPy{for} loop
\end{itemize}
%
\begin{codebox}[Treat Bankrupt Players]
\begin{minted}[fontsize=\scriptsize, linenos]{python}
def kickBrokePlayers (players) :
    for player in players :
        if player['money'] <= 0 :
            player['state'] = 'retired'
\end{minted}
\end{codebox}
%
\end{frame}

% =========================================================================== %

\begin{frame}[fragile]{Analysis / Synthesis: End-Of-Game-Screen}
%
\begin{itemize}
\item Sort all players by total money
	\begin{itemize}
	\item \texttt{sort} does this very conveniently
	\end{itemize}
\item Print their names and money on screen
	\begin{itemize}
	\item Simple \inPy{for} loop
	\end{itemize}
\item Call after Main Game Loop
\end{itemize}
%
\begin{codebox}[End-Of-Game-Screen]
\begin{minted}[fontsize=\scriptsize, linenos]{python}
def endOfGame(players) :
    players.sort(key = lambda player : player['money'])
    
    for player in players :
        print(f"{player['name']:30}|{player['money']:9.2f}")
\end{minted}
\end{codebox}
%
\end{frame}

% =========================================================================== %

\begin{frame}{The Game}
%
\begin{center}
	\begin{large}
		\emph{You probably just lost The Game.}\\
		\emph{(I did.)}
	\end{large}
	
	\vspace{20pt}
	\url{https://en.wikipedia.org/wiki/The_Game_(mind_game)}
\end{center}
%
\end{frame}