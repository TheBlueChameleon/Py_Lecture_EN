% =========================================================================== %

\begin{frame}[t,plain]
\titlepage
\end{frame}

% =========================================================================== %

\begin{frame}{Recap}
%
\begin{columns}[T]
\column{.5\linewidth}
\begin{itemize}
\item Classes
	\begin{itemize}
	\item Data collection with context
	\item Class attributes and instance attributes
	\item Methods: Functions that access specific instances
	\end{itemize}
\end{itemize}
%
\column{.5\linewidth}
\begin{itemize}
\item Dunders: Methods, called automatically in specific situations
	\begin{itemize}
	\item \inPy{__init__} -- Constructor
	\item \inPy{__str__} and \inPy{__repr__} -- text representation
	\item \inPy{__call__} -- instance as function
	\end{itemize}
\end{itemize}

\end{columns}
%
\begin{center}
	\emph{Any Questions?}
\end{center}
%
\end{frame}

% =========================================================================== %

\begin{frame}[fragile]{Recap}
%
\begin{columns}[T]
\column{.5\linewidth}
\begin{itemize}
\item \inPy{__len__} and \inPy{__abs__}
\item Comparison operators
\item Computation operators
	\begin{itemize}
	\item Left- and right hand sided operators
	\item Return value \inPy{NotImplemented}
	\end{itemize}
\item Index-Access
\item Type safety
\end{itemize}
%
\column{.5\linewidth}
\begin{itemize}
\item Design-Pattern: Shared methods
\item Inheritance
	\begin{itemize}
	\item \enquote{Recyle} classes
	\item Methods and attributes are copied
	\item Everything can be overwritten
	\item Calling Parent-Methods -- \inPy{super()}
	\end{itemize}
\end{itemize}

\end{columns}
%
\begin{center}
	\emph{Any Questions?}
\end{center}
%
\end{frame}

% =========================================================================== %

\begin{frame}[fragile]{Seen in the Exercises}
%
\begin{tcbraster}[raster columns=2,
                  raster equal height,
                  nobeforeafter,
                  raster column skip=0.5cm]
\begin{warnbox}[Example: Interupted Blocks, leftupper=6mm]
\begin{minted}[fontsize=\scriptsize, linenos]{python}
class foo :
    def __init__(self, attr) :
        self.bar = attr

foobar = foo("foo bar")

    def __str__(self) :
        return str(self.bar)

print(foobar)
\end{minted}
\end{warnbox}
%
\begin{codebox}[Example: Solution]
\begin{minted}[fontsize=\scriptsize, linenos]{python}
class foo :
    def __init__(self, attr) :
        self.bar = attr

    def __str__(self) :
        return str(self.bar)

foobar = foo("foo bar")
print(foobar)
\end{minted}
\end{codebox}
\end{tcbraster}
%
\begin{hintbox}[Hint]
This holds for \emph{all} block structures: loops, functions, \inPy{if}s, \ldots
\end{hintbox}
%
\end{frame}

% =========================================================================== %

\begin{frame}[fragile]{Mind-Pictures: Classes as Professions}
%
\begin{itemize}
\item Class instance: \enquote{Person, holding some specific knowledge}
	\begin{itemize}
	\item Pure knowledge \thus instance attributes
	\item Expert of their subject, can interpret these attributes \thus methods
	\end{itemize}
\item Designing a class: \enquote{describing the profession}
	\begin{itemize}
	\item What do all people of that profession know? \thus instance and class attributes
	\item What questions can they answer? \thus methods
	\item How can they alter their \enquote{knowledge}? \thus methods, too
	\end{itemize}
\item Interaction between classes
	\begin{itemize}
	\item Picture: conversation
	\item One person (instance) will take the lead, asks the other one questions
	\item Ask questions \thus call methods
	\item For that we need to know:
		\begin{itemize}
		\item Whom to ask \thus instance of the other class
		\item Which question \thus which method
		\item Possibly: How to specify the question \thus arguments to the method
		\end{itemize}
	\end{itemize}
\end{itemize}
%
\end{frame}

% =========================================================================== %

\begin{frame}{Mind-Pictures: Classes as Professions}
%
\begin{itemize}
\item Example: \emph{Where, in a given garden, is it a good idea to plant a tree?}
\item For this: two \enquote{experts}
	\begin{itemize}
	\item Gardener: knows how to recognize a good spot
	\item Owner of the garden: knows the shape of their garde
	\end{itemize}
\item Possible: two approaches
	\begin{itemize}
	\item Responsibility with gardener
		\begin{itemize}
		\item Asks question: which spots are there? Decides on a good spot based on this reply
		\end{itemize}
	\item Responsibility with owner
		\begin{itemize}
		\item Asks if a chosen spot is a good spot
		\end{itemize}
	\end{itemize}
\item (Third approach: gardener and owner are the same person)
	\begin{itemize}
	\item Less complex mechanics for \emph{simple} examples
	\item Big projects: danger to loose track
	\item \enquote{A class is responsible for \emph{one} concept.}
	\end{itemize}
\end{itemize}
%
\end{frame}

% =========================================================================== %

\begin{frame}[fragile]
%
\begin{tcbraster}[raster columns=2,
                  raster equal height,
                  nobeforeafter,
                  raster column skip=0.5cm]
\begin{codebox}[Example: Respons. w. Gardener]
\begin{minted}[fontsize=\scriptsize, linenos]{python}
import random

class Owner :
  def __init__ (self) :
    self._spots = [1, 2, 3, 4, 5]
  
  def getSpots (self) :
    return self._spots

class Gardener :
  def getGoodSpot(self, whomToAsk) :
    answer = []
    for spot in whomToAsk.getSpots() :
      answer.append(spot)
    return random.choice(answer)

g = Gardener()
o = Owner()

print( g.getGoodSpot(o) )
\end{minted}
\end{codebox}
%
\begin{codebox}[Example: Responsibilty with Owner]
\begin{minted}[fontsize=\scriptsize, linenos]{python}
import random

class Gardener :
  def isSpotGood(self, spot) :
    return random.choice([True, False])

class Owner :
  def __init__(self) :
    self._spots = [1, 2, 3, 4, 5]
  
  def getGoodSpot(self, whomToAsk) :
    goodSpots = []
    for spot in self._spots :
      if whomToAsk.isSpotGood(spot) :
        goodSpots.append(spot)
    return random.choice(goodSpots)
    
g = Gardener()
o = Owner()

print( o.getGoodSpot(g) )
\end{minted}
\end{codebox}
\end{tcbraster}
%
\end{frame}

% =========================================================================== %

\begin{frame}
%
\begin{hintbox}[Whom to give the responsibilty?]
Both approaches make sense. Which one is better depends on the form of the project as a whole. Some thoughts that go into the decision:
\begin{itemize}
\item Which approach minimized the number of function calls (\thus efficiency)
\item Which object will be used mostly by the end-user (\thus user interface)
\item Which object already controls most of the required data/methods (\thus hierarchy)
\end{itemize}

These questions can lead to contradictory results; there isn't always one ulitimative answer.
\end{hintbox}
%
\end{frame}

% =========================================================================== %

\begin{frame}[fragile]{Chapter 9}
%
\begin{itemize}
\item \inPy{try..except} blocks
\item Multipartite \inPy{except} blocks
\item \inPy{else} and \inPy{finally}
\item \inPy{raise} -- causing exceptions
\item Creating own exception classes
\end{itemize}
%
\end{frame}

% =========================================================================== %

\begin{frame}[fragile]{\inPy{try..except} Blocks -- Basics}
%
\begin{columns}[T]
\column{.5\linewidth}
\begin{itemize}
\item Sometimes easier to \emph{react} to errors than to catch them before they happen
\item Code structure:
	\begin{itemize}
	\item \inPy{try}: anything that could go wrong
	\item \inPy{except}: what to do when something has gone wrong
	\end{itemize}
\item Discern \emph{error classes}, aka \emph{exeptions}
\item After treatment of this exception: program continues normally
\end{itemize}
%
\column{.5\linewidth}
\begin{codebox}[Syntax: \texttt{try..except} Block]
\begin{minted}[fontsize=\scriptsize]{python}
try :
    Code that could raise an exception
except ErrorClass as varible :
    Code for error treatment
\end{minted}
\end{codebox}
\end{columns}
%
\end{frame}

% =========================================================================== %

\begin{frame}[fragile]
%
\begin{tcbraster}[raster columns=2,
                  raster equal height,
                  nobeforeafter,
                  raster column skip=0.5cm]
\begin{codebox}[Example: Code w. Error Treatment (1)]
\begin{minted}[linenos, fontsize=\scriptsize]{python}
import math

for x in range(-3, 4) :
  try :
    print( 1/x,  math.log(abs(x-2)) )
  except ZeroDivisionError as e :
    print("Error:", e)
\end{minted}
\end{codebox}
%
\begin{cmdbox}[Output: Code w. Error Treatment (1)]
\begin{minted}[fontsize=\scriptsize]{text}
-0.3333333333333333 1.6094379124341003
-0.5 1.3862943611198906
-1.0 1.0986122886681098
Error: division by zero
1.0 0.0
Traceback (most recent call last):
  File "<stdin>", line 3, in <module>
ValueError: math domain error
\end{minted}
\end{cmdbox}
\end{tcbraster}
%
\begin{center}
\begin{itemize}
\item[\Thus] Execution \emph{jumps} upon encountering an exception into the \inPy{except} block.
\item[\Thus] Thereafter: code continues normally
\item[\Thus] Only specified error classes are caught
	\begin{itemize}
	\item[\thus] Here: \inPy{ZeroDivisionError}
	\end{itemize}
\item[\Thus] Untreated errors (here: \inPy{ValueError}) still lead to aborting execution
\item[\Thus] Variable \texttt{e} holds \enquote{Description of the error}
	\begin{itemize}
	\item[\thus] Instance of \inPy{ZeroDivisionError}
	\end{itemize}
\end{itemize}
\end{center}
%
\end{frame}

% =========================================================================== %

\begin{frame}[fragile]
%
\begin{tcbraster}[raster columns=2,
                  raster equal height,
                  nobeforeafter,
                  raster column skip=0.5cm]
\begin{codebox}[Example: Code w. Error Treatment (2)]
\begin{minted}[linenos, fontsize=\scriptsize]{python}
try :
    for x in range(-3, 4) :
        print(1 / x)
except ZeroDivisionError as e :
    print("Error:", e)

print("Done.")
\end{minted}
\end{codebox}
%
\begin{cmdbox}[Output: Code w. Error Treatment (2)]
\begin{minted}[fontsize=\scriptsize]{text}
-0.3333333333333333
-0.5
-1.0
Error: division by zero
Done.
\end{minted}
\end{cmdbox}
\end{tcbraster}
%
\begin{center}
\begin{itemize}
\item[\Thus] \emph{Entire} \inPy{try} block is left after the first exception; continuation of code \emph{after} the \inPy{try..except} block
\item[\Thus] Hierarchical ordering of \inPy{try} and other structures is important!
\end{itemize}
\end{center}
%
\end{frame}

% =========================================================================== %

\begin{frame}[fragile]{\inPy{tuple}s of Error Classes in a \inPy{except} Block}
%
\begin{columns}[T]
\column{.5\linewidth}
\begin{itemize}
\item In one \inPy{try} block there can be multiply things going wrong
\item Simple case: React to all errors in the same way
\item If so: list error classes as a \inPy{tuple}
\item[\Thus] If \emph{any} of the error classes meets the exception, treatment is triggered
\end{itemize}
%
\column{.5\linewidth}
\begin{codebox}[Syntax: \texttt{try..except} Block]
\begin{minted}[fontsize=\scriptsize]{python}
try :
    Code that might cause an exception
except (class1, class2, ..) as varible :
    Code for treatment
\end{minted}
\end{codebox}
\end{columns}
%
\end{frame}

% =========================================================================== %

\begin{frame}[fragile]
%
\begin{tcbraster}[raster columns=2,
                  raster equal height,
                  nobeforeafter,
                  raster column skip=0.5cm]
\begin{codebox}[Example: Code w. Error Treatment (3)]
\begin{minted}[linenos, fontsize=\scriptsize]{python}
import math

for x in range(-3, 4) :
    try :
        print(1/x,  math.log(abs(x-2)))
        
    except (ZeroDivisionError,
            ValueError) as e :
        print("Error:", e)
\end{minted}
\end{codebox}
%
\begin{cmdbox}[Output: Code w. Error Treatment (3)]
\begin{minted}[fontsize=\scriptsize]{text}
-0.3333333333333333 1.6094379124341003
-0.5 1.3862943611198906
-1.0 1.0986122886681098
Error: division by zero
1.0 0.0
Error: math domain error
0.3333333333333333 0.0
\end{minted}
\end{cmdbox}
\end{tcbraster}
%
\end{frame}

% =========================================================================== %

\begin{frame}[fragile]{Multipartite \inPy{except} Blocks}
%
\begin{columns}[T]
\column{.5\linewidth}
\begin{itemize}
\item Sometimes necessary: treat different errors differently
	\begin{itemize}
	\item Invalid entry in a file? Skip
	\item File not found? Ask for alternative name
	\end{itemize}
\item Solution: Multiple \inPy{except} blocks
\end{itemize}
%
\column{.5\linewidth}
\begin{codebox}[Syntax: \texttt{try..except}-Block]
\begin{minted}[fontsize=\scriptsize]{python}
try :
    Code that may raise an exception
except Class1 as varible :
    Code for treatment of Class1
except Class2 as varible :
    Code for treatment of Class2
except ...
\end{minted}
\end{codebox}
\end{columns}
%
\end{frame}

% =========================================================================== %

\begin{frame}[fragile]{Hierarchical Order of Error Classes}
%
\begin{itemize}
\item Error Classes have hierarchical order: parent classes and derived classes
\item \enquote{X belongs to Y}
\item Example: \inPy{ZeroDivisionError} is a \inPy{ArithmeticError}, too
\item But: \inPy{ArithmeticError} is not necessarily a \inPy{ZeroDivisionError}
\item[\Thus] \inPy{except ArithmeticError} catches a \inPy{ZeroDivisionError}, but not vice versa
\item \eg \inPy{ArithmeticError} also covers \inPy{OverflowError}
\item See lecture notes or \url{https://docs.python.org/3/library/exceptions.html} for a list of pre-defined error classes
\item \inPy{Exception} covers \emph{all} possible errors
\end{itemize}
%
\end{frame}

% =========================================================================== %

\begin{frame}[fragile]
%
\begin{tcbraster}[raster columns=2,
                  raster equal height,
                  nobeforeafter,
                  raster column skip=0.5cm]
\begin{codebox}[Example: Multiple Error Classes A]
\begin{minted}[linenos, fontsize=\scriptsize]{python}
try :
    print(1/0)
    
except ZeroDivisionError as e :
    print("ZeroDivisionError")
    
except ArithmeticError as e :
    print("ArithmeticError")
\end{minted}
\end{codebox}
%
\begin{codebox}[Example: Multiple Error Classes B]
\begin{minted}[linenos, fontsize=\scriptsize]{python}
try :
    print(1/0)
    
except ArithmeticError as e :
    print("ArithmeticError")
    
except ZeroDivisionError as e :
    print("ZeroDivisionError")
\end{minted}
\end{codebox}
\end{tcbraster}
%
\begin{tcbraster}[raster columns=2,
                  raster equal height,
                  nobeforeafter,
                  raster column skip=0.5cm]
\begin{cmdbox}[Output: Multiple Error Classes A]
\begin{minted}[fontsize=\scriptsize]{text}
ZeroDivisionError
\end{minted}
\end{cmdbox}
%
\begin{cmdbox}[Output: Multiple Error Classes B]
\begin{minted}[fontsize=\scriptsize]{text}
ArithmeticError
\end{minted}
\end{cmdbox}
\end{tcbraster}
%
\begin{itemize}
\item[\Thus] Only the first matching block is executed
\item[\Thus] Other \inPy{except} blocks -- even if matching the error -- are ignored
\end{itemize}
%
\end{frame}

% =========================================================================== %

\begin{frame}[fragile]{\inPy{else} and \inPy{finally} with \inPy{try..except} Blocks}
%
\begin{columns}[T]
\column{.5\linewidth}
\begin{itemize}
\item \inPy{else}: execute if \emph{no error} has occured
\item \enquote{On success of the \inPy{try} block, do ...}
\end{itemize}
%
\column{.5\linewidth}
\begin{itemize}
\item \inPy{finally}: Execution in any case
\item Even if there is a non-treated error
\item \enquote{Safe Shutdown}
\item Also executed after an \inPy{else} block (if present)
\end{itemize}
\end{columns}
%
\end{frame}

% =========================================================================== %

\begin{frame}[fragile]
%
\begin{tcbraster}[raster columns=2,
                  raster equal height,
                  nobeforeafter,
                  raster column skip=0.5cm]
\begin{codebox}[Example: All Clauses]
\begin{minted}[linenos, fontsize=\scriptsize]{python}
def divide(x, y):
    try:
        result = x / y
    except ZeroDivisionError:
        print("division by zero!")
    else:
        print("result is", result)
    finally:
        print("finally clause")

print("divide(2, 1)")
divide(2, 1)
print()

print("divide(2, 0)")
divide(2, 0)
print()

print('divide("2", "0")')
divide("2", "1")
\end{minted}
\end{codebox}
%
\begin{cmdbox}[Output: All Clauses]
\begin{minted}[fontsize=\scriptsize]{text}
divide(2, 1)
result is 2.0
finally clause

divide(2, 0)
division by zero!
finally clause

divide("2", "1")
finally clause
Traceback (most recent call last):
  File "<stdin>", line 1, in <module>
  File "<stdin>", line 3, in divide
TypeError: unsupported operand type(s) 
  for /: 'str' and 'str'
\end{minted}
\end{cmdbox}
\end{tcbraster}
%
\end{frame}

% =========================================================================== %

\begin{frame}[fragile]{Triggering Exceptions: \inPy{raise}}
%
\begin{columns}[T]
\column{.5\linewidth}
\begin{itemize}
\item Wrong usage of a class or function may cause harm
\item Example: User enters a file name \Thus could overwrite existing files
\item Syntactically correct, but undesired behaviour
\item[\Thus] \emph{Throw/Raise an exception}
\end{itemize}
%
\column{.5\linewidth}
\begin{codebox}[Syntax: \texttt{raise}]
\begin{minted}[fontsize=\scriptsize]{python}
raise ErrorClass(Arguments)
\end{minted}
\end{codebox}
%
\begin{itemize}
\item \texttt{ErrorClass}: As shown before: \inPy{ArithmeticError}, \inPy{ZeroDivisionError}, ...
\item \texttt{Arguments}: Depends on Error Class
	\begin{itemize}
	\item Usually: String, will be output in Error Handling
	\end{itemize}
\end{itemize}
\end{columns}
%
\end{frame}

% =========================================================================== %

\begin{frame}[fragile]
%
\begin{tcbraster}[raster columns=2,
                  raster equal height,
                  nobeforeafter,
                  raster column skip=0.5cm]
\begin{codebox}[Example: Raising Exceptions]
\begin{minted}[linenos, fontsize=\scriptsize]{python}
print("Please choose an option:")
print("START game")
print("show HIGHSCORE")
print("QUIT")

option = input("> ")

if not option.upper() in [
    "START",
    "HIGHSCORE",
    "QUIT"
] :
    raise RuntimeError(
        "Invalid choice"
    )
\end{minted}
\end{codebox}
%
\begin{cmdbox}[Output: Raising Exceptions]
\begin{minted}[fontsize=\scriptsize]{text}
Please choose an option:
START game
show HIGHSCORE
QUIT
> SETTINGS
Traceback (most recent call last):
  File "<stdin>", line 2, in <module>
RuntimeError: Invalid choice
\end{minted}
\end{cmdbox}
\end{tcbraster}
%
\end{frame}

% =========================================================================== %

\begin{frame}[fragile]{Re-\inPy{raise}}
%
\begin{columns}[T]
\column{.5\linewidth}
\begin{itemize}
\item Possibly desired: Debug-Output, but no full treatment of the exception
\item Superior functional unit (end user) treats the problem
\item[\Thus] Reactivate error after \enquote{treatment}: \texttt{raise} \emph{without} error class
\end{itemize}
%
\vspace{15pt}
\begin{cmdbox}[Ausgabe: re-\texttt{raise}]
\begin{minted}[fontsize=\scriptsize]{text}
division by zero
x = 1, y = 0
Error was caught, eventually
\end{minted}
\end{cmdbox}
%
\column{.5\linewidth}
\begin{codebox}[Beispiel: re-\texttt{raise}]
\begin{minted}[fontsize=\scriptsize]{python}
def divide(x, y) :
  try :
    result = x / y
  except ZeroDivisionError as e:
    print(e)
    print(f"x = {x}, y = {y}")
    raise
  else :
    return result

try :
  print( divide(1, 0) )
except ZeroDivisionError as e:
  print("Error was caught, eventually")
\end{minted}
\end{codebox}
%
\end{columns}
%
\end{frame}

% =========================================================================== %

\begin{frame}[fragile]{}
%
\begin{hintbox}[Choose Error Classes with Maximum Specificity]
Error classes should be chosen very narrowly, such that treatment can be written effectively. 

Catching Errors too unspecifically can cause more harm than doing good.
\end{hintbox}
%
\end{frame}

% =========================================================================== %

\begin{frame}[fragile]{Creating Own Error Classes}
%
\begin{columns}[T]
\column{.5\linewidth}
\begin{itemize}
\item Error Classes are, in fact \emph{classes} in the Python sense
\item Hierarchical Structure through \emph{inheritance}
\item All necessary properties can be inherited from class \inPy{Exception} (or from classes derived thereof)
\item Class may remain empty otherwise (\inPy{pass}), or hold custom code deemed useful for error treatment
\end{itemize}
%
\column{.5\linewidth}
\begin{codebox}[Syntax: Creating an Own Error Class]
\begin{minted}[fontsize=\scriptsize]{python3}
class OwnErrorClass (Exception) :
   pass
\end{minted}
\end{codebox}
%
\begin{itemize}
\item In \inPy{Exception}: String with a short summary (\inPy{print(e)})
\item You may use your full knowledge about classes: attributes, methods, dunders, ...
\end{itemize}
%
\end{columns}
%
\end{frame}

% =========================================================================== %

\begin{frame}[fragile]
%
\begin{codebox}[Example: Creating an Own Error Class]
\begin{minted}[linenos,fontsize=\scriptsize]{python3}
class UserInputError(Exception) :
    pass


try:
    x = float(input("Please provide a positive number"))
    if x < 0 :
        raise UserInputError("Negative number entered!")

except UserInputError as e :
    print(e)
\end{minted}
\end{codebox}
%
\end{frame}